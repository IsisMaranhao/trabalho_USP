% Options for packages loaded elsewhere
\PassOptionsToPackage{unicode}{hyperref}
\PassOptionsToPackage{hyphens}{url}
%
\documentclass[
]{article}
\title{PIXAR}
\author{Isis Maranhão}
\date{22/02/2022}

\usepackage{amsmath,amssymb}
\usepackage{lmodern}
\usepackage{iftex}
\ifPDFTeX
  \usepackage[T1]{fontenc}
  \usepackage[utf8]{inputenc}
  \usepackage{textcomp} % provide euro and other symbols
\else % if luatex or xetex
  \usepackage{unicode-math}
  \defaultfontfeatures{Scale=MatchLowercase}
  \defaultfontfeatures[\rmfamily]{Ligatures=TeX,Scale=1}
\fi
% Use upquote if available, for straight quotes in verbatim environments
\IfFileExists{upquote.sty}{\usepackage{upquote}}{}
\IfFileExists{microtype.sty}{% use microtype if available
  \usepackage[]{microtype}
  \UseMicrotypeSet[protrusion]{basicmath} % disable protrusion for tt fonts
}{}
\makeatletter
\@ifundefined{KOMAClassName}{% if non-KOMA class
  \IfFileExists{parskip.sty}{%
    \usepackage{parskip}
  }{% else
    \setlength{\parindent}{0pt}
    \setlength{\parskip}{6pt plus 2pt minus 1pt}}
}{% if KOMA class
  \KOMAoptions{parskip=half}}
\makeatother
\usepackage{xcolor}
\IfFileExists{xurl.sty}{\usepackage{xurl}}{} % add URL line breaks if available
\IfFileExists{bookmark.sty}{\usepackage{bookmark}}{\usepackage{hyperref}}
\hypersetup{
  pdftitle={PIXAR},
  pdfauthor={Isis Maranhão},
  hidelinks,
  pdfcreator={LaTeX via pandoc}}
\urlstyle{same} % disable monospaced font for URLs
\usepackage[margin=1in]{geometry}
\usepackage{color}
\usepackage{fancyvrb}
\newcommand{\VerbBar}{|}
\newcommand{\VERB}{\Verb[commandchars=\\\{\}]}
\DefineVerbatimEnvironment{Highlighting}{Verbatim}{commandchars=\\\{\}}
% Add ',fontsize=\small' for more characters per line
\usepackage{framed}
\definecolor{shadecolor}{RGB}{248,248,248}
\newenvironment{Shaded}{\begin{snugshade}}{\end{snugshade}}
\newcommand{\AlertTok}[1]{\textcolor[rgb]{0.94,0.16,0.16}{#1}}
\newcommand{\AnnotationTok}[1]{\textcolor[rgb]{0.56,0.35,0.01}{\textbf{\textit{#1}}}}
\newcommand{\AttributeTok}[1]{\textcolor[rgb]{0.77,0.63,0.00}{#1}}
\newcommand{\BaseNTok}[1]{\textcolor[rgb]{0.00,0.00,0.81}{#1}}
\newcommand{\BuiltInTok}[1]{#1}
\newcommand{\CharTok}[1]{\textcolor[rgb]{0.31,0.60,0.02}{#1}}
\newcommand{\CommentTok}[1]{\textcolor[rgb]{0.56,0.35,0.01}{\textit{#1}}}
\newcommand{\CommentVarTok}[1]{\textcolor[rgb]{0.56,0.35,0.01}{\textbf{\textit{#1}}}}
\newcommand{\ConstantTok}[1]{\textcolor[rgb]{0.00,0.00,0.00}{#1}}
\newcommand{\ControlFlowTok}[1]{\textcolor[rgb]{0.13,0.29,0.53}{\textbf{#1}}}
\newcommand{\DataTypeTok}[1]{\textcolor[rgb]{0.13,0.29,0.53}{#1}}
\newcommand{\DecValTok}[1]{\textcolor[rgb]{0.00,0.00,0.81}{#1}}
\newcommand{\DocumentationTok}[1]{\textcolor[rgb]{0.56,0.35,0.01}{\textbf{\textit{#1}}}}
\newcommand{\ErrorTok}[1]{\textcolor[rgb]{0.64,0.00,0.00}{\textbf{#1}}}
\newcommand{\ExtensionTok}[1]{#1}
\newcommand{\FloatTok}[1]{\textcolor[rgb]{0.00,0.00,0.81}{#1}}
\newcommand{\FunctionTok}[1]{\textcolor[rgb]{0.00,0.00,0.00}{#1}}
\newcommand{\ImportTok}[1]{#1}
\newcommand{\InformationTok}[1]{\textcolor[rgb]{0.56,0.35,0.01}{\textbf{\textit{#1}}}}
\newcommand{\KeywordTok}[1]{\textcolor[rgb]{0.13,0.29,0.53}{\textbf{#1}}}
\newcommand{\NormalTok}[1]{#1}
\newcommand{\OperatorTok}[1]{\textcolor[rgb]{0.81,0.36,0.00}{\textbf{#1}}}
\newcommand{\OtherTok}[1]{\textcolor[rgb]{0.56,0.35,0.01}{#1}}
\newcommand{\PreprocessorTok}[1]{\textcolor[rgb]{0.56,0.35,0.01}{\textit{#1}}}
\newcommand{\RegionMarkerTok}[1]{#1}
\newcommand{\SpecialCharTok}[1]{\textcolor[rgb]{0.00,0.00,0.00}{#1}}
\newcommand{\SpecialStringTok}[1]{\textcolor[rgb]{0.31,0.60,0.02}{#1}}
\newcommand{\StringTok}[1]{\textcolor[rgb]{0.31,0.60,0.02}{#1}}
\newcommand{\VariableTok}[1]{\textcolor[rgb]{0.00,0.00,0.00}{#1}}
\newcommand{\VerbatimStringTok}[1]{\textcolor[rgb]{0.31,0.60,0.02}{#1}}
\newcommand{\WarningTok}[1]{\textcolor[rgb]{0.56,0.35,0.01}{\textbf{\textit{#1}}}}
\usepackage{graphicx}
\makeatletter
\def\maxwidth{\ifdim\Gin@nat@width>\linewidth\linewidth\else\Gin@nat@width\fi}
\def\maxheight{\ifdim\Gin@nat@height>\textheight\textheight\else\Gin@nat@height\fi}
\makeatother
% Scale images if necessary, so that they will not overflow the page
% margins by default, and it is still possible to overwrite the defaults
% using explicit options in \includegraphics[width, height, ...]{}
\setkeys{Gin}{width=\maxwidth,height=\maxheight,keepaspectratio}
% Set default figure placement to htbp
\makeatletter
\def\fps@figure{htbp}
\makeatother
\setlength{\emergencystretch}{3em} % prevent overfull lines
\providecommand{\tightlist}{%
  \setlength{\itemsep}{0pt}\setlength{\parskip}{0pt}}
\setcounter{secnumdepth}{-\maxdimen} % remove section numbering
\usepackage{booktabs}
\usepackage{longtable}
\usepackage{array}
\usepackage{multirow}
\usepackage{wrapfig}
\usepackage{float}
\usepackage{colortbl}
\usepackage{pdflscape}
\usepackage{tabu}
\usepackage{threeparttable}
\usepackage{threeparttablex}
\usepackage[normalem]{ulem}
\usepackage{makecell}
\usepackage{xcolor}
\ifLuaTeX
  \usepackage{selnolig}  % disable illegal ligatures
\fi

\begin{document}
\maketitle

\hypertarget{introduuxe7uxe3o}{%
\section{\texorpdfstring{\textbf{Introdução}:}{Introdução:}}\label{introduuxe7uxe3o}}

\emph{Pixar Animation Studios}, conhecida mundialmente como Pixar, foi
fundada em 03 de fevereiro de 1986.

Inicialmente, o estúdio era uma divisão da Lucasfilm. Recebeu o
financiamento do co-fundador da Apple, Steve Jobs, que se tornou seu
acionista majoritário. Algumas referência dizem que o começo de sua
história está igualmente relacionado a um clipe animado de uma mão
esquerda, a um dos criadores da Apple, e claro, à própria The Walt
Disney Company.

Para quem tiver a mesma curiosidade, esse é o link para o vídeo referido
acima. Está disponível no youtube.

\href{https://www.youtube.com/watch?v=wdedV81UQ5k\&t=22s}{Vídeo:
https://www.youtube.com/watch?v=wdedV81UQ5k\&t=22s}

A Walt Disney anunciou em 2006 a aquisição da Pixar Animation Studios
por US\$ 7,4 bilhões em ações. Esse negócio buscava restaurar a
importância da Disney no segmento de animação infantil e levar Steve
Jobs, presidente da Pixar e da Apple da época, a uma posição de destaque
na Disney.

\hypertarget{luxo-jr.}{%
\subsection{Luxo Jr.}\label{luxo-jr.}}

\emph{Não sabe nada sobre Luxo jr?}

\begin{figure}
\centering
\includegraphics{https://i1.wp.com/cinemaepixels.com.br/wp-content/uploads/2018/10/luxo-jr-anima\%C3\%A7\%C3\%A3o-pixar.jpg}
\caption{Luxo Jr.}
\end{figure}

Luxo Jr.~foi a segunda curta-metragem produzida pela Pixar Animation
Studios. Foi um filme de curta-metragem de animação feito com computação
gráfica em 1986.

Ele pode ser visto esmagando o ``I'' da Pixar no início de cada filme
produzido pelo estúdio de animação.

\begin{figure}
\centering
\includegraphics{https://br.web.img2.acsta.net/c_310_420/medias/nmedia/18/92/03/98/20176570.jpg}
\caption{Luxo Jr.: outubro de 2021}
\end{figure}

No link abaixo você pode ver o vídeo mais atual desse curta no
\emph{Youtube}.

\url{https://www.youtube.com/watch?v=FI0T0Oj7WFE}

\hypertarget{as-premiauxe7uxf5es-de-luxo-jr.}{%
\subsubsection{As premiações de Luxo
Jr.~}\label{as-premiauxe7uxf5es-de-luxo-jr.}}

E brilhante o quanto esse curta pode mostrar que é possível passar
sentimentos através de uma animação em 3D, mesmo se o personagem for uma
lâmpada.

Por conta desse extraordinário feito recebeu os seguintes prêmios e
indicações:

O Luxo Jr.~não ficou só nesse curta, ele também protagonizou mais
curtas-metragens de animação feitos com computação gráfica em 1990 pela
Pixar Animation Studios. São eles:

\begin{itemize}
\tightlist
\item
  Leve e Pesado (Light \& Heavy)
\item
  Surpresa (Surprise)
\item
  Em Cima e em Baixo (Up and Down)
\item
  Em Frente e Atrás (Front and Back)
\end{itemize}

\textbf{PRÊMIOS DO FILME LUXO JR.}

id

ANO

PRÊMIO

CATEGORIA

RESULTADO

{1}

1986

Oscar

Melhor Curta Animado

Apenas indicado

{2}

1987

Festival Internacional de Filmes em Berlin

Melhor Filme de Curta-metragem

Ganhou

{3}

1987

Festival Internacional de Animação em Ottawa

Melhor filme com menos de 5 minutos

Ganhou em 2° lugar

{4}

1987

Celebração Mundial de Animação

Animação Assistida de Computador

Ganhou

\hypertarget{dados-da-pixar}{%
\section{\texorpdfstring{\textbf{Dados da
Pixar}:}{Dados da Pixar:}}\label{dados-da-pixar}}

Usando os dados dispóníveis no \emph{pacote dados} que foram traduzidos
e disponibilizados numa versão em português.

\href{https://cienciadedatos.github.io/dados/}{Dados: Filmes Pixar}

É fato que a empresa é refênrencia quando estamos falando sobre filmes
de animação. A quatidade de prêmios que a \textbf{PIXAR} ganhou, com os
seus filmes mais conhecidos pelo grande público, estão no gráfico
abaixo.

\includegraphics{PIXAR_files/figure-latex/unnamed-chunk-9-1.pdf}

Mas para fazer uma análise da significância dos números dessa empresa,
vamos usar os dados da tabela abaixo, onde podemos associar os números
de outros prêmios importantes do mundo do cinema com os valores
imponentes dos orçamentos e bilheteria mundial de cada filme.\^{}

Olhando para a base de dados acima, podemos usar a \emph{função do R:
summary()}. Essa função calcula um resumo estatístico dos dados e
objetos de uma base de dados.

\begin{verbatim}
##     filme           nota_rotten_tomatoes nota_metacritic bilheteria_mundial 
##  Length:23          Min.   : 40.00       Min.   :57.00   Min.   :1.354e+08  
##  Class :character   1st Qu.: 84.00       1st Qu.:71.00   1st Qu.:4.230e+08  
##  Mode  :character   Median : 96.00       Median :81.00   Median :6.237e+08  
##                     Mean   : 89.17       Mean   :79.96   Mean   :6.358e+08  
##                     3rd Qu.: 97.50       3rd Qu.:90.00   3rd Qu.:8.323e+08  
##                     Max.   :100.00       Max.   :96.00   Max.   :1.243e+09  
##    orcamento        
##  Min.   : 30000000  
##  1st Qu.:120000000  
##  Median :175000000  
##  Mean   :156565217  
##  3rd Qu.:192500000  
##  Max.   :200000000
\end{verbatim}

Como uma forma simples de aplicar matemática nos dados da \emph{PIXAR},
vamos aplicar um modelo de regressão linear para avaliar a relação entre
as duas vaiáveis quantititivas, no nosso caso, bilheteria\_mundial e
orcamento dos filmes.

Sem entrar muito na teoria matemática, queremos encontrar um modelo que
descreva a seguinte exressão

\[
y = a + bx
\]

Abaixo é possível ver os resultados da \emph{função do R: summary()} ao
resultado. Nesse resultado tem vários pontos inportantes

\begin{verbatim}
## 
## Call:
## lm(formula = bilheteria_mundial ~ orcamento, data = selecao_1)
## 
## Residuals:
##        Min         1Q     Median         3Q        Max 
## -539565806 -166309089    6823841  157837088  514654078 
## 
## Coefficients:
##              Estimate Std. Error t value Pr(>|t|)
## (Intercept) 3.030e+08  2.123e+08   1.427    0.168
## orcamento   2.126e+00  1.302e+00   1.633    0.117
## 
## Residual standard error: 285300000 on 21 degrees of freedom
## Multiple R-squared:  0.1127, Adjusted R-squared:  0.07046 
## F-statistic: 2.668 on 1 and 21 DF,  p-value: 0.1173
\end{verbatim}

\begin{Shaded}
\begin{Highlighting}[]
\NormalTok{R2 }\OtherTok{\textless{}{-}}\NormalTok{ (}\FunctionTok{sum}\NormalTok{((selecao\_1}\SpecialCharTok{$}\NormalTok{yhat }\SpecialCharTok{{-}} \FunctionTok{mean}\NormalTok{(selecao\_1}\SpecialCharTok{$}\NormalTok{orcamento))}\SpecialCharTok{\^{}}\DecValTok{2}\NormalTok{))}\SpecialCharTok{/}
\NormalTok{  ((}\FunctionTok{sum}\NormalTok{((selecao\_1}\SpecialCharTok{$}\NormalTok{yhat }\SpecialCharTok{{-}} \FunctionTok{mean}\NormalTok{(selecao\_1}\SpecialCharTok{$}\NormalTok{orcamento))}\SpecialCharTok{\^{}}\DecValTok{2}\NormalTok{)) }\SpecialCharTok{+}\NormalTok{ (}\FunctionTok{sum}\NormalTok{((selecao\_1}\SpecialCharTok{$}\NormalTok{erro)}\SpecialCharTok{\^{}}\DecValTok{2}\NormalTok{)))}

\FunctionTok{round}\NormalTok{(R2, }\AttributeTok{digits =} \DecValTok{4}\NormalTok{)}
\end{Highlighting}
\end{Shaded}

\begin{verbatim}
## [1] 0.7629
\end{verbatim}

Para quem não entende muito bem a matemática que estamos trabalhando, um
R² = 0.7629 significa que o modelo linear explica 76,29\% da variância
da variável dependente a partir do regressores (variáveis independentes)
incluídas naquele modelo linear.

\includegraphics{PIXAR_files/figure-latex/unnamed-chunk-19-1.pdf}

O Teste de Shapiro-Wilk tem como objetivo avaliar se uma distribuição é
semelhante a uma distribuição normal.

Como resultado, o teste retornará a estatística W, que terá um valor de
significância associada, o valor-p.~Para dizer que uma distribuição é
normal, o valor p precisa ser maior do que 0,05.

\begin{verbatim}
## 
##  Shapiro-Francia normality test
## 
## data:  modelo_filme$residuals
## W = 0.98036, p-value = 0.8444
\end{verbatim}

\end{document}
